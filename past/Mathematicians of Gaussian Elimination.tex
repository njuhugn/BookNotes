\documentclass{article}
\usepackage{amsmath}
\usepackage{color}

\begin{document}
\title{Gauss Elimination}
\author{Guangneng~Hu}
\maketitle

\section{Gauss Elimination}
\textcolor{blue}{\emph{Mathematicians of Gaussian Elimination} by Joseph F. Grcar 2011}
\begin{enumerate}
\item 
\textcolor{red}{The sole development in ancient times was in China. }
\begin{itemize}
\item \textbf{\emph{Jiuzhang Suanshu, or The Nine Chapters on the Mathematical Art}}
\item \textbf{Liu Hui}
\end{itemize}

\item
\textcolor{red}{An independent origin in modern Europe has had three phases.}
\begin{itemize}
\item First came the schoolbook lesson, beginning with \textbf{Isaac Newton}.
\item Next were methods for professional hand computers, which began with \textbf{Carl Friedrich Gauss}, who apparently was inspired by work of \textbf{Joseph Louis Lagrange}.
\item Last was the interpretation in matrix algebra by several authors, including \textbf{John von Neumann}.
\end{itemize}

\item
\textcolor{red}{Coda:~An \emph{algorithm} is a series of steps for solving a
mathematical problem}
\begin{itemize}
\item John Todd (1911-2007) lectured on Jensen��s work in 1946.
\item Alan Turing (1912-1954) described it in the way universities now teach it.
\item George Forsythe (1917-1972) called it ��Gaussian elimination��.
\end{itemize}

\end{enumerate}

\textcolor{blue}{There may be no other subject that has been molded by so many renowned mathematicians.}

\begin{equation*}
     \left( \begin{array}{cc}
    1 & 2\\
    3 & 4\end{array} \right)
    %%%
     = \left( \begin{array}{cc}
    1/3 & 1\\
    1   & 0\end{array} \right)
    %%%
    \left( \begin{array}{cc}
    3 & 4\\
    0 & 2/3\end{array} \right)
\end{equation*}

\end{document} 